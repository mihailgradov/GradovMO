\documentclass[12pt]{article}
\usepackage[utf8]{inputenc}
\usepackage[russian]{babel}
\usepackage{amsmath,amssymb}
\usepackage{graphics}
\usepackage{graphicx}
\graphicspath{{pictures/}}
\DeclareGraphicsExtensions{.pdf,.png,.jpg}
\usepackage{pbox}
\usepackage[x11names]{xcolor}
\definecolor{brightmaroon}{rgb}{0.76, 0.13, 0.28}
\definecolor{royalazure}{rgb}{0.0, 0.22, 0.66}
\usepackage[colorlinks=true,linkcolor=royalazure]{hyperref}
\usepackage{tikz, tkz-fct, pgfplots}
\usetikzlibrary{arrows}
\usepackage{geometry}
\geometry{
	a4paper,
	total={170mm,257mm},
	left=20mm,
	top=20mm
} 
\usepackage[labelsep=period]{caption}


% ----------------- Commands ----------------- 
\newcommand{\eps}{\varepsilon}
\newcommand\tline[2]{$\underset{\text{#1}}{\text{\underline{\hspace{#2}}}}$}

% ----------------- Set graphics path ----------------- 
\graphicspath{{img/}}

\begin{document}
	\pagestyle{empty}
	\centerline{\large Министерство науки и высшего образования}	
	\centerline{\large Федеральное государственное бюджетное образовательное}
	\centerline{\large учреждение высшего образования}
	\centerline{\large ``Московский государственный технический университет}
	\centerline{\large имени Н.Э. Баумана}
	\centerline{\large (национальный исследовательский университет)''}
	\centerline{\large (МГТУ им. Н.Э. Баумана)}
	\hrule
	\vspace{0.5cm}
	\begin{figure}[h]
		\center{\includegraphics[scale=0.35]{bmstu-logo-small.png}}
	\end{figure}
	\begin{center}
		\large	
		\begin{tabular}{c}
			Факультет ``Фундаментальные науки'' \\
			Кафедра ``Высшая математика''		
		\end{tabular}
	\end{center}
	\vspace{0.5cm}
	\begin{center}
		\LARGE \bf	
		\begin{tabular}{c}
			\textsc{Отчёт} \\
			по учебной практике \\
			за 3 семестр 2020---2021 гг.
		\end{tabular}
	\end{center}
	\vspace{0.5cm}
	\begin{center}
		\large
		\begin{tabular}{p{5.3cm}ll}
			\pbox{5.45cm}{
				Руководитель практики,\\
				ст. преп. кафедры ФН1} 	& \tline{\it(подпись)}{5cm} & Кравченко О.В. \\[0.5cm]
			студент группы ФН1--31 		& \tline{\it(подпись)}{5cm} & Градов М.О.
		\end{tabular}
	\end{center}
	\vfill
	\begin{center}
		\large	
		\begin{tabular}{c}
			Москва, \\
			2020 г.
		\end{tabular}
	\end{center}
	\newpage	
	\tableofcontents
	
	\newpage
	\section{Цели и задачи практики}	
	\subsection{Цели}
	--- развитие компетенций, способствующих успешному освоению материала бакалавриата и необходимых в будущей профессиональной деятельности.
	
	\subsection{Задачи}
	\begin{enumerate}
		\item Знакомство с теорией рядов Фурье, и теорией интегральный уравнений.
		\item Развитие умения поиска необходимой информации в специальной литературе и других источниках.
		\item Развитие навыков составления отчётов и презентации результатов.
	\end{enumerate}
	
	\subsection{Индивидуальное задание}	
	\begin{enumerate}
		\item Изучить способы отображения математической информации в системе вёртски \LaTeX.
		\item Изучить возможности  системы контроля версий \textsf{Git}.
		\item Научиться верстать математические тексты, содержащие формулы и графики в системе \LaTeX.
		Для этого, выполнить установку свободно распространяемого дистрибутива \textsf{TeXLive} и оболочки \textsf{TeXStudio}.
		\item Оформить в системе \LaTeX типовые расчёты по курсу математического анализа согласно своему варианту.
		\item Создать аккаунт на онлайн ресурсе \textsf{GitHub} и загрузить исходные \textsf{tex}--файлы 
		и результат компиляции в формате \textsf{pdf}.
		\item Решить индивидуальное домашнее задание согласно своему варианту, и оформить решение с учётов пп. 1---4.
	\end{enumerate} 
	
	\newpage
	\section{Отчёт}
	Интегральные уравнения имеют большое прикладное значение, являясь мощным
	орудием исследования многих задач естествознания и техники: они широко используются
	в механике, астрономии, физике, во многих задачах химии и биологии. Теория линейных
	интегральных уравнений представляет собой важный раздел современной математики,
	имеющий широкие приложения в теории дифференциальных уравнений, математической
	физике, в задачах естествознания и техники. Отсюда владение методами теории
	дифференциальных и интегральных уравнений необходимо прикладному математику, при решении задач
	механики и физики.
	
	\newpage
	\section{Индивидуальное задание}

	
	%=================================================================================================================================
	\subsection{Ряды Фурье и интегральное уравнение Вольтерры.}
%	
	
	%=================================================================================================================================

% ---------------------------- Problem 1----------------------------------
\subsubsection*{\center Задача № 1.}
{\bf Условие.~}
Разложить в ряд Фурье заданную функцию $f(x)$, построить графики $f(x)$ и суммы ее ряда Фурье. Если не указывается, какой вид разложения в ряд необходимо представить, то требуется разложить функцию либо в общий тригонометрический ряд Фурье, либо следует выбрать оптимальный вид разложения в зависимости от данной функции.


\begin{equation}
	f(x) = \cos{x}, \quad 0 \leqslant x \leqslant \pi \quad\text{по синусам кратных дуг}.
\end{equation}

{\bf Решение.~}	
\noindent
В общем случае ряд Фурье имеет вид
$$
f(x)=\frac{a_0}{2}+\sum_{n=1}^\infty 
\left(a_n\cos{(nx)}+b_n\sin{(nx)}\right).
$$
\noindent
Вычислим коэффициенты
$$
\begin{array}{rcl}
	a_n &=& 0,\quad\text{где}\ n = 0,1,...											\\[12pt]
	b_n &=& \displaystyle\frac{2}{T}\left(
	\int\limits_0^T
	f(x) \sin \frac{\pi nx}{T}\,dx \right) \quad T=\pi \quad \Rightarrow \quad b_n = \displaystyle\frac{2}{\pi}\left(
	\int\limits_0^\pi
	\cos{x} \sin{nx} \,dx \right).
\end{array}
$$
\noindent
Решим
$$
\begin{array}{lrc}
\displaystyle \int \cos{x} \sin{nx} \,dx = \displaystyle \int \sin{nx} \,d\sin{x} = \displaystyle \sin{x}\sin{nx} - \\[12pt] - \int \sin{x} \,d\sin{nx} = \displaystyle \sin{nx}\sin{x} - n\int \sin{x}\cos{nx} \,dx 
 = \displaystyle \sin{x}\sin{nx} + n\int \cos{nx} \,d\cos{x} = \\[12pt] = \displaystyle \sin{x}\sin{nx} +  n\cos{x}\cos{nx} - n\int \cos{x} \,d\cos{nx} = \displaystyle \sin{x}\sin{nx} + n\cos{x}\cos{nx} + n^2\int \cos{x}\sin{nx} \,dx. 
\end{array}
$$
\noindent
Имеем
$$
\begin{array}{lrc}
	\displaystyle (1-n^2)\cdot\int \cos{x} \sin{nx} \,dx = \displaystyle \sin{x}\sin{ns} + n\cos{x}\cos{nx} \Leftrightarrow  \displaystyle \int \cos{x}\sin{nx} = \\[12pt] = \displaystyle \frac{\sin{x}\sin{nx}+n\cos{x}\cos{nx}}{1-n^2}.
\end{array}
$$
\noindent
При этом данное выражение не имеет смысла при n = 1
$$
\begin{array}{lrc}
	\displaystyle \int\limits_0^\pi \cos{x}\sin{x} \,dx = \displaystyle \frac{1}{(1-n)^{2}}\left(\sin{x}\sin{nx} + n\cos{x}\cos{nx}\right)\bigg|_0^\pi = \displaystyle \frac{1}{1-n^2}(\sin{\pi}\sin{n\pi} + n\cos{\pi}\cos{n\pi} -
	\\[12pt] -\sin(0)\sin(0) - n\cos(0)\cos(0) ) = \displaystyle \frac{1}{1-n^2}\left(- n\cos{n\pi}- n  \right) = \displaystyle \frac{1}{1-n^2}\left((-1)^{n+1} -1 \right) . 
\end{array}
$$
\noindent
Отдельно для n=1
$$
\begin{array}{lrc}
	 \displaystyle \int\limits_0^\pi \cos{x}\sin{x} \,dx = \displaystyle \int\limits_0^\pi \sin{x} \,d\sin{x} = \displaystyle \frac{1}{2}\sin^2{x} \bigg|_0^\pi = 0
\end{array}
$$ 
\noindent
Тогда получаем
$$
\begin{array}{lrc}
		b_n &=& \frac{2}{\pi} \cdot \frac{n}{1-n^2}\left( (-1)^{n+1}-1\right) \quad\text{при}\ n \geq 2 \quad\text{и}\quad b_n=0 \quad\text{при} \quad n = 1 (n\in \mathbb {N})
\end{array}
$$ 
График функции $S(x)$ имеет следующий вид
\begin{center}
	\begin{tikzpicture}
		\begin{axis}[xmin=-10, xmax=10, ymin=-3, ymax=3,
			width=0.8\textwidth,
			height=0.4\textwidth,
			axis x line=middle,
			axis y line=middle, 
			every axis x label/.style={at={(current axis.right of origin)},anchor=west},
			every inner x axis line/.append style={|-latex'},
			every inner y axis line/.append style={|-latex'},
			minor tick num=1,			
			axis equal=true,
			xlabel=$x$, 
			ylabel=$S(x)$,          
			samples=100,
			clip=true,
			]
			\addplot[color=black, line width=1.5pt, domain=0:3.14] {cos(deg(x))};
	     	\addplot[color=black, line width=1.5pt,domain=6.28:9.42] {cos(deg(x))};
		    \addplot[color=black, line width=1.5pt,domain=-3.14:-6.28]{cos(deg(x))};
		
			\addplot[color=black, line width=1.5pt,domain=3.14:6.28]{sin(deg(x-1.57))};
			\addplot[color=black, line width=1.5pt,domain=0:-3.14]{sin(deg(x+4.71))};
			\addplot[color=black, line width=1.5pt,domain=-6.28:-9.42] {sin(deg(x+10.99))};
			
			\addplot[thick,dashed] coordinates {(-3.14, 0) (-3.14, -1)};
			\addplot[thick,dashed] coordinates {(3.14, 0) (3.14, 1)};
			\addplot[thick,dashed] coordinates {(-3.14, 0) (-3.14, 1)};
			\addplot[thick,dashed] coordinates {(3.14, 0) (3.14, -1)};
			
			\addplot[thick,dashed] coordinates {(-6.28, 0) (-6.28, 1)};
			\addplot[thick,dashed] coordinates {(-6.28, 0) (-6.28, -1)};
			\addplot[thick,dashed] coordinates {(6.28, 0) (6.28, 1)};
			\addplot[thick,dashed] coordinates {(6.28, 0) (6.28, -1)};
			
			\addplot[thick,dashed] coordinates {(9.42, 0) (9.42, 1)};
			\addplot[thick,dashed] coordinates {(9.42, 0) (9.42, -1)};
			\addplot[thick,dashed] coordinates {(-9.42, 0) (-9.42, -1)};
			\addplot[thick,dashed] coordinates {(-9.42, 0) (-9.42, 1)};
		
			\addplot[
			mark=*,
			mark options={fill=black, draw=black},
			only marks,
			] coordinates {(-9.42, 0) (-6.28, 0) (-3.14, 0) (0, 0) (3.14, 0) (6.28, 0) (9.42, 0)};
		\end{axis}
	\end{tikzpicture}
\end{center}
\noindent
\textbf{Ответ:}
\[ 
\begin{split} 
	&f(x)=\sum_{n=2}^\infty\left[\frac{2}{\pi}\cdot\frac{n}{1-n^2}\left( (-1)^{n+1}-1\right)\sin{nx}\right] = \frac{2}{\pi}\cdot \sum_{n=2}^\infty\left[\frac{n}{1-n^2}\left((-1)^{n+1}-1\right)\sin{nx}\right].
\end{split} 
\]




% ---------------------------- Problem 2----------------------------------
\subsubsection*{\center Задача № 2.}
{\bf Условие.~}
Для заданной графически функции $y(x)$ построить ряд Фурье в комплексной форме, изобразить график суммы построенного ряда

\[
f(x)=\begin{cases}
	\cos{x}, 	& 0 \leqslant x \leqslant \frac{\pi}{2},\\
	0, 	& \frac{\pi}{2} \leqslant x \leqslant \pi.
\end{cases}
\]
\noindent
\textbf{Решение.}\\

\noindent
Ряд Фурье в комплексной форме имеет следующий вид
\[
f(x) = \sum_{n=-\infty}^\infty c_n e^{i\omega nx},\quad c_n=\frac{1}{T}\int\limits_a^b f(x) e^{-i\omega nx}dx,\,\omega=\frac{2\pi}{T}.
\]
В нашем примере $ a=0,b=\pi,T=\pi,\omega=2$,
найдем коэффицинеты $c_n,\,n=0,\pm1,\pm2,\ldots$.
$$
\begin{array}{rcl}
	c_n &=&\displaystyle\frac{1}{\pi}\left(
	\int\limits_0^\pi f(x) e^{-2i nx}dx \right) = \displaystyle\frac{1}{\pi}\left(
	\int\limits_0^\frac{\pi}{2} \cos{x} e^{-2i nx}dx \right) + \displaystyle\frac{1}{\pi}\left(
	\int\limits_\frac{\pi}{2}^\pi 0 \cdot e^{-2i nx}dx \right)  = \\[12pt]
	&=&\displaystyle\frac{1}{\pi} \left(\int\limits_0^\frac{\pi}{2} \cos{x} e^{-2i nx}dx  \right).
\end{array}
$$
\noindent
Решим
$$
\begin{array}{lrc}
\displaystyle \int \cos{x} e^{-2i nx}dx = \displaystyle -\frac{1}{2in} \left(\int\cos{x} de^{-2i nx}\right) = \displaystyle -\frac{1}{2in} \left(\int\cos{x} e^{-2i nx}\right) + \frac{1}{2in} \left(\int e^{-2i nx} d\cos{x} \right) = \\[12pt]
	= \displaystyle -\frac{1}{2in} \left(\int\cos{x} e^{-2i nx}\right) - \frac{1}{2in} \left(\int e^{-2i nx} d\sin{x} \right) = \displaystyle -\frac{1}{2in} \left(\int\cos{x} e^{-2i nx}\right) 
	- \frac{1}{4n^2} \left(\int \sin{x}  de^{-2i nx} \right) = \\[12pt]
	= \displaystyle -\frac{1}{2in} \left(\cos{x} e^{-2i nx}\right)	- \frac{1}{4n^2} \left( \sin{x}  e^{-2i nx} \right)  \displaystyle +\frac{1}{4n^2} \left(\int e^{-2i nx} \cos{x} dx \right).
\end{array}
$$
\noindent
Имеем
$$
\begin{array}{lrc}
	\displaystyle \int \cos{x} e^{-2i nx}dx = \displaystyle -\frac{1}{2in} \left(\cos{x} e^{-2i nx}\right)	- \frac{1}{4n^2} \left( \sin{x}  e^{-2i nx} \right)  \displaystyle +\frac{1}{4n^2} \left(\int e^{-2i nx} \cos{x} dx \right)  \Leftrightarrow 
	\\[12pt] \Leftrightarrow  \displaystyle \int \cos{x} e^{-2i nx} dx = \frac{2in\cos{x}e^{-2inx}-\sin{x}e^{-2inx}}{4n^{2}} = \frac{e^{-2inx}(2in\cos{x}-\sin{x})}{4n^{2}-1} .
\end{array}
$$  
\noindent
Данный интеграл не определен при $4n^2-1=0 \Leftrightarrow n= \pm 	\displaystyle \frac{1}{2} \notin \mathbb{Z}$.
Таким образом, 
 $$
\begin{array}{lrc}
	 	\displaystyle \int\limits_0^\frac{\pi}{2} \cos{x} e^{-2i nx} dx = \frac{e^{-2inx}(2in\cos{x}-\sin{x})}{4n^{2}-1} = -\frac{e^{-i\pi x} + 2in}{4n^{2}-1}
\end{array}
$$

\noindent
На основании теоремы Дирихле построими график:
\begin{center}
	\begin{tikzpicture}
		\begin{axis}[xmin=-6, xmax=6, ymin=-1, ymax=0.5,
			width=0.8\textwidth,
			height=0.4\textwidth,
			axis x line=middle,
			axis y line=middle, 
			every axis x label/.style={at={(current axis.right of origin)},anchor=west},
			every inner x axis line/.append style={|-latex'},
			every inner y axis line/.append style={|-latex'},
			minor tick num=1,			
			axis equal=true,
			xlabel=$x$, 
			ylabel=$S(x)$,          
			samples=100,
			clip=true,
			]
			
			\addplot[color=black, line width=1.5pt,domain=1.57:3.14] {0};
			\addplot[color=black, line width=1.5pt,domain=0:1.57]{cos(deg(x))};
			\addplot[color=black, line width=1.5pt,domain=3.14:4.71] {sin(deg(x-1.57))};
			\addplot[color=black, line width=1.5pt,domain=4.71:6.28] {0};
			\addplot[color=black, line width=1.5pt,domain=-1.57:0] {0};
			\addplot[color=black, line width=1.5pt,domain=-1.57:-3.14]{sin(deg(x+4.71))};
			\addplot[color=black, line width=1.5pt,domain=-4.71:-3.14] {0};
			\addplot[color=black, line width=1.5pt,domain=-6.28:-4.71]{cos(deg(x))};
			 
			\addplot[thick,dashed] coordinates {(-6.28,0) (-6.28,1)};
			\addplot[thick,dashed] coordinates {(-3.14,0) (-3.14,1)};
			\addplot[thick,dashed] coordinates {(3.14,0) (3.14,1)};
			\addplot[
			mark=*,
			mark options={fill=black, draw=black},
			only marks,
			] coordinates {(-6.28, 0.5) (-3.14, 0.5) (0, 0.5) (3.14, 0.5)};
		\end{axis}
	\end{tikzpicture}
\end{center}

\noindent
\textbf{Ответ:}
\[
\begin{split}
	&f(x)= \sum_{n=-\infty}^\infty\left[-\frac{1}{\pi} \cdot \frac{e^{-i\pi n} + 2in}{4n^{2}-1} e^{2inx}\right] = -\frac{1}{\pi}\sum_{n=-\infty}^\infty\left[-\frac{e^{-i\pi n} + 2in}{4n^{2}-1} e^{2inx}\right]; \\[12pt]
   & \quad\text{при этом} \ f(x) \ \text{для} \ x \ \text{вида}\ \pi n, ~n\in\mathbb{Z} \ \text{равно} \ \frac{1}{2}.
\end{split}
\]



% ---------------------------- Problem 3----------------------------------
\subsubsection*{\center Задача № 3.}
{\bf Условие.~}\\
Найти резольвенту для интегрального уравнения Вольтерры со следующим ядром 
$$ K(x,t)= \displaystyle x^{\frac{1}{8}}t^{\frac{1}{4}}.$$

\noindent
{\bf Решение.~}\\
\noindent
$$
\begin{array}{lrc}
	K_1(x,t)=\displaystyle x^{\frac{1}{8}}t^{\frac{1}{4}}, \\[12pt]
	K_2(x,t)=\displaystyle\int\limits_t^x x^{\frac{1}{8}}s^{\frac{1}{4}} \cdot s^{\frac{1}{8}}t^{\frac{1}{4}} = \displaystyle x^{\frac{1}{8}}t^{\frac{1}{4}} \int\limits_t^x s^{\frac{3}{8}} ds = \displaystyle\frac{8}{11} x^{\frac{1}{8}}t^{\frac{1}{4}}\left(x^{\frac{11}{8}} - t^{\frac{11}{8}} \right) = \displaystyle \frac{8}{11}K_1(x,t)\left(x^{\frac{11}{8}} - t^{\frac{11}{8}} \right), \\[12pt]
\end{array}
$$	
$$
\begin{array}{lrc}
K_3(x,t)=\displaystyle\int\limits_t^x K(x,s)K_2(s,t)ds = \displaystyle \int\limits_t^x K(x,s)\cdot \frac{8}{11}K_1(s,t)\left(x^{\frac{11}{8}} - t^{\frac{11}{8}}  \right) \,ds = 
	\\[12pt] = \displaystyle \frac{8}{11} \int\limits_t^x s^{\frac{11}{8}} K(x,s) K_1(s,t) \,ds - 
- \frac{8}{11} \int\limits_t^x t^{\frac{11}{8}} K(x,s) K_1(s,t) \,ds = \\[12pt] =  \displaystyle \frac{8}{11} \int\limits_t^x s^{\frac{11}{8}} K(x,s) K_1(s,t) \,ds - \frac{8}{11} t^{\frac{11}{8}}  K_2(x,t) .\\[12pt]
\text{Решим} \ \displaystyle\int\limits_t^x s^{\frac{8}{11}} K(x,s)K_1(s,t) \,ds = 	\displaystyle\int\limits_t^x s^{\frac{8}{11}} x^{\frac{1}{4}} s^{\frac{1}{4}} s^{\frac{1}{8}} t^{\frac{1}{4}} \,ds = K_1(x,t)\int\limits_t^x s^{\frac{14}{8}} \,ds =  K_1(x,t)\frac{s^{22/8}}{22/8} = \\[12pt] = K_1(x,t) : \frac{8}{22}(x^{22/8} - t^{22/8}) \\[12pt]

\text{Значит:} \ \displaystyle K_3(x,t) =  \frac{8}{11} \cdot \frac{8}{22} K_1(x,t) (x^{22/8} - t^{22/8}) -  \frac{8}{11} t^{11/8} K_2(x,t).
\\[12pt]

\text{Аналогичными вычислениями показывается, что} \\[12pt]
K_4(x,t) = \displaystyle \frac{8}{11} \cdot \frac{8}{22} \cdot \frac{8}{33} K_1(x,t)(x^{33/8} - t^{33/8}) - \frac{8}{22} \cdot \frac{8}{22} t^{11/8} K_2(x,t) - \frac{8}{11} t^{\frac{11}{8}} K_3(x,t) \\[12pt]
K_5(x,t) = \displaystyle\frac{8}{11} \cdot \frac{8}{22} \cdot \frac{8}{33} \cdot \frac{8}{44} K_1(x,t)(x^{44/8} - t^{44/8}) -\frac{8}{11} \cdot \frac{8}{22} \cdot \frac{8}{33}t^{33/8} K_2(x,t) - \frac{8}{11} \cdot \frac{8}{22}K_3(x,t) - \\[12pt] - \displaystyle  \frac{8}{11} \cdot t^{\frac{11}{8}} K_1(x,t) 
\\[12pt]
\text{Легко видеть, что} \\
	K_j(x,t)=\displaystyle\frac{8^{j-1}}{\displaystyle \prod_{n=1}^{j-k}11n} K_1(x,t)\left( x^{\frac{11(j-1)}{8}} - t^{\frac{11(j-1)}{8}} \right) - \sum_{k=2}^{j-1}\left[\displaystyle K_k(x,t) t^{\frac{11(j-k)}{8}} \cdot \frac{8^{j-k}}{\displaystyle \prod_{n=1}^{j-k}11n} \right] 
\end{array}
$$
Теперь можно получить
$$ 
R(x,t,\lambda)=\sum_{p=1}^\infty \lambda^{p-1 }K_p(x,t).
$$
	\newpage
\addcontentsline{toc}{section}{Список литературы}
\begin{thebibliography}{99}
	\bibitem{book01} Львовский С.М. Набор и вёрстка в системе \LaTeX,\,2003.
	\bibitem{book02} Краснов М.Л., Киселев А.И., Макаренко Г.И. Интегральные уравнения. М.:~Наука,\,1976.
	\bibitem{book03} Васильева А. Б., Тихонов Н. А. Интегральные уравнения. --- 2-е изд., стереотип. --- М:~ФИЗМАТЛИТ,\,2002.
\end{thebibliography}

\end{document}